% Copyright (C) 2004  Daniel Wagenaar (wagenaar@caltech.edu)
%
% This program is free software; you can redistribute it and/or modify
% it under the terms of the GNU General Public License as published by
% the Free Software Foundation; either version 2 of the License, or
% (at your option) any later version.
%
% This program is distributed in the hope that it will be useful,
% but WITHOUT ANY WARRANTY; without even the implied warranty of
% MERCHANTABILITY or FITNESS FOR A PARTICULAR PURPOSE.  See the
% GNU General Public License for more details.
%
% You should have received a copy of the GNU General Public License
% along with this program; if not, write to the Free Software
% Foundation, Inc., 59 Temple Place, Suite 330, Boston, MA  02111-1307  USA

\documentclass{article}
\usepackage{txfonts}
\usepackage{sectsty}
\usepackage{listings}
\lstset{language=matlab,basicstyle=\ttfamily\small}
\sectionfont{\bf\normalsize}
\usepackage[letterpaper,width=7in,height=9in]{geometry}
\begin{document}
\noindent{\Large\bf Multixburst --- 
a burst detector based on individual channels}
\medskip

\noindent{By Daniel Wagenaar, 7/28/2003}


\section{Introduction}

The new burst detector works by first finding all bursts on individual
channels, then combines those into barrages by linking together those
individual bursts that overlap in time. To avoid confusion, I'll use
the phrase {\it simplex burst} for a burst on an individual channel. A
simplex burst is not necessarily a true single channel event: it may
well co-occur with simplex bursts on other channels. When simplex
bursts are collected together, the result is a {\it multix burst}.
Multix bursts may be dishwide barrages, or more localized events,
even including single channel bursts.

I'll use words in preference to maths whenever possible in the
following text, but a few symbols help keep the explanation coherent:

\begin{itemize}
\item[$f_c$] is the average firing rate on channel $c$, that is, the total
number of spikes recorded on that channel divided by the duration of
the recording.
\item[$\tau_c$] is the threshold inter-spike interval (ISI) for
channel $c$. For each channel, $\tau_c$ is set to the smaller of
$1\over{4f_c}$ and 100~ms. The factor {\bf four\footnote{Most of the
parameters of the algorithm are tunable. Rather than assigning symbols to
all parameters, I'll typeset them in bold, to signify that they can be
changed. The given values are ``sensible defaults''.}} ensures that
only spikes that succeed each other faster than four times the average
firing rate, can be considered bursts.
\end{itemize}



\section{Finding simplex bursts}

The algorithm considers each channel fully independently. For a given
channel, it searches for sequences of {\bf four} or more spikes with
all included ISIs less than $\tau_c$. After these `core' bursts have
been found, they are extended into the past and the future to also
contain other spikes that have ISIs less than {\bf 200~ms}, (or less
than $1\over{3f_c}$, whichever is smaller). Thus, a simplex burst must
consist of a core of at least very four very closely spaced spikes,
with an `entourage' of any number of slightly less closely spaced
spikes.

\section{Combining simplex bursts into multix bursts}\label{multix}

Once all simplex bursts on all channels have been found, they are
sorted in temporal order. A multix burst is simply a sequence of one
or more simplex bursts that have any amount of overlap. Note that if a
channel fires a burst all by itself, that event will still be detected
as a multix burst, irrespective of the fact that it consists of only
one simplex burst.

\section{Post-processing}

In many cultures, a small number of channels keep firing at strongly
elevated rates for extended periods, sometimes until the next barrage.
If that happens, several barrages would all be globbed together into one
multix burst according to the algorithm in section \ref{multix}. This
problem is corrected in a post-processing stage called {\it
disjoinxburst}. It looks at each detected multix burst in turn, and
constructs a graph of number of simultaneous simplex bursts vs time.
If a multix burst corresponds to several barrages, this graph will
have more than one hump. The algorithm finds these humps, and splits
the multix bursts accordingly.

\section{Finding superbursts}

I'm not yet very satisfied with the way superbursts are detected,
because there are too many parameters to twiddle. The basic operation
is as follows:

\begin{itemize}
\item Find all `big' bursts, i.e., multix bursts with at least half of
all `good' channels participating. (A `good' channel is a channel that
participates in at least three bursts over the course of the
recording.)

\item Find sequences of {\bf two} or more big bursts that have internal
inter-burst intervals (IBIs) less than {\bf 5~s}, and preceding and
succeeding IBIs larger than {\bf three} times the average internal
IBI (AIIBI). This defines the core superburst. 

\item Add to this core any multix bursts (big or small) that happen
within the core superburst, or within one AIIBI of the core.
\end{itemize}

This seems to work fine on recordings that have visibly clear
superbursts, but it's not very elegant. Also, in recordings were there
are bursts of many different sizes, which events are elevated to
superburst status appears to be somewhat arbitrary.

\section{Using the code}

Here is an example of the use of the {\it multix} package.

\begin{quotation}~\vspace{-\baselineskip}
\begin{lstlisting}{}
spk=loadspike_noc('/d4/dw/somefile.spike',2); % Load a spike file
tms=spk.time; chs=spk.channel; % Extract times and channel numbers
bb=simplexburst(tms,chs,.1,4,.25,3,4,5);  % Find simplex bursts
    % Here's what those parameters mean:
    %  .1:  100 ms threshold ISI
    %  4:   minimum 4 spikes in the core burst
    %  .25: 250 ms threshold ISI for entourage
    %  3:   min. 3x elevated firing rate in entourage
    %  4:   min. 4x elevated firing rate in core
    %  5:   min. 5 spikes in burst total

mm=multixburst(bb); % Find the candidate multix bursts
mm=disjoinxburst(mm,bb); % ... and post-process them to disjoin globbed ones

kk=bigxburst(mm); % Find all big multix bursts
ss=superxburst(mm,kk,10,1); % Use that to find core superbursts
    % 10 means: no IBIs more than 10s inside a superburst core
    % 1 means: use mean onset/offset times of component simplex bursts rather
    %          than first onset/last offset.

ess=expandxburst(ss,mm,kk,1); % Add entourage to superbursts
\end{lstlisting}\end{quotation}

\section{Implementation}

The file multix-1.0.tgz contains the implementation in matlab, as well
as a digital version of this document. Refer to the matlab code for
precise usage information. Each function outputs a structure with
several burst parameters. Type, e.g., `help multixbursts' to learn
what to expect from {\it multixbursts.m}.

\end{document}
